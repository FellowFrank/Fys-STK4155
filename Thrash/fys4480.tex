\documentclass[aspectratio=169]{beamer}

%--- Theme & Colors ---
\usetheme{Madrid}
\usecolortheme{beaver}
\setbeamertemplate{navigation symbols}{}
\setbeamertemplate{caption}[numbered]

%--- Packages ---
\usepackage{amsmath, amssymb, amsfonts}
\usepackage{physics} % For bra-ket notation \ket{}, \bra{}
\usepackage{graphicx}
\usepackage{booktabs} % For nice tables
\usepackage{tikz}     % For drawing diagrams
\usetikzlibrary{shapes,arrows,positioning}

%--- Meta Data ---
\title[Many-Body Methods]{Computational Many-Body Methods in Physics}
\subtitle{}
\author{Anton N. Torgersen}
\institute[UiO]{University of Oslo \\ FYS4480}
\date{\today}

\begin{document}

%=== Title Slide ===
\begin{frame}
    \titlepage
\end{frame}

%=== Outline ===
\begin{frame}{Outline}
    \tableofcontents
\end{frame}

%===========================================================
\section{Introduction and Formalism}
%===========================================================

\begin{frame}{Introduction: The Many-Body Problem}
    \textbf{Goal:} Solve the Schrödinger equation for $N$ interacting particles.
    $$ \hat{H} \ket{\Psi} = E \ket{\Psi} $$
    \textbf{The Challenge:} The Hilbert space grows factorially with $N$-particles and $K$-orbitals. A change in one particle creates a corresponding change in the entire many-body state.\\
    $\implies$ Exact diagonalization (FCI) is impossible for large systems.
    

\end{frame}





\begin{frame}{The Independent Particle Basis}
    \textbf{The Pauli Principle}
    Fermionic wavefunctions must be anti-symmetric. The simplest ansatz satisfying this is the \textbf{Slater Determinant}:
    
    $$ \ket{\Phi_0} \leftrightarrow \Psi_{SD}(\mathbf{r}_1, \dots, \mathbf{r}_N) = \frac{1}{\sqrt{N!}} \det \left| \phi_p(\mathbf{r}_q) \right| $$
    

    \textbf{Second Quantization}
    To handle this efficiently, we work in Fock space using creation ($a^\dagger$) and annihilation ($a$) operators.
    
    \begin{itemize}
        \item \textbf{ correspondence:} A Slater Determinant is represented as a string of operators acting on the vacuum:
        $$ \ket{\Phi_0} = \prod_{i \le F} a_i^\dagger \ket{0} = a_1^\dagger a_2^\dagger \dots a_N^\dagger \ket{0} $$
        \item \textbf{Anti-commutation:} $\{a_p, a_q^\dagger\} = \delta_{pq}$ enforces the Pauli principle.
    \end{itemize}
\end{frame}

\begin{frame}{The Model System: Pairing Model}
    To compare methods, we use the \textbf{Pairing Hamiltonian} as a test case:
    
    $$ \hat{H} = \hat{H}_0 + \hat{H}_I $$
    
    \begin{itemize}
        \item \textbf{Single Particle ($\hat{H}_0$):} simple energy levels with equal spacing.
        $$\hat{H}_0 = \sum_{p,\sigma} (p-1) \hat{a}_{p \sigma}^{\dagger} \hat{a}_{p \sigma}$$
        \item \textbf{Interaction ($\hat{H}_I$):} Constant pairing strength $g$. Moves pairs $(p+, p-)$ to $(q+, q-)$.
        $$\hat{H}_I = -\frac{1}{2} g \sum_{pq} \hat{a}_{p+}^{\dagger} \hat{a}_{p-}^{\dagger} \hat{a}_{q-} \hat{a}_{q+}$$
    \end{itemize}
\end{frame}

\begin{frame}{Pairing model representation}
We will truncate the Hilbert space to only have 4 particles and 4 levels.
\begin{figure}
    \centering
    \includegraphics[width=0.6\textwidth]{fys4480-Pairing.png}
    \caption{Schematic of the pairing model}
    \label{fig:pairing_model}
\end{figure}
\end{frame}

%===========================================================
\section{Full Configuration Interaction (FCI)}
%===========================================================

\begin{frame}{Full Configuration Interaction (FCI)}
    \textbf{The Exact Solution}
    \begin{itemize}
        \item Expands the wavefunction $\Psi$ in the complete basis of all possible Slater Determinants generated from the reference $\ket{\Phi_0}$.
        \item \textbf{The Expansion:}
        $$ \ket{\Psi_{FCI}} = \left( 1 + \sum_{ia} \hat{C}_i^a + \sum_{ijab} \hat{C}_{ij}^{ab} + \dots \right) \ket{\Phi_0} $$
        $$ \ket{\Psi_{FCI}} = c_0 \ket{\Phi_0} + \sum_{ia} c_i^a \ket{\Phi_i^a} + \sum_{ijab} c_{ij}^{ab} \ket{\Phi_{ij}^{ab}} + \dots $$
    \end{itemize}

    \textbf{The Method: Exact Diagonalization}
    \begin{itemize}
        \item Construct the full Hamiltonian matrix in this many-body basis.
        \item Solve the eigenvalue problem $\hat{H}\mathbf{c} = E\mathbf{c}$ to find exact energies and coefficients.
    \end{itemize}
\end{frame}

\begin{frame}{FCI Scaling}
    \textbf{The Curse of Dimensionality}
    The dimension of the Hilbert space ($D_{FCI}$) is the number of ways to distribute $N$ particles in $K$ orbitals:
    $$ D_{FCI} = \binom{K}{N} = \frac{K!}{N!(K-N)!} $$
    \begin{itemize}
        \item Scales factorially with system size.
        \item FCI is only feasible for very small systems
    \end{itemize}
\end{frame}


\begin{frame}{Pairing Matrix Elements}
    \textbf{The Hamiltonian Matrix} for $N=4$ particles in $K=4$ levels:
    \[
            H = \begin{pmatrix}
            2-g & -g/2 & -g/2 & -g/2 & -g/2 & 0 \\
            -g/2 & 4-g & -g/2 & -g/2 & 0 & -g/2 \\
            -g/2 & -g/2 & 6-g & 0 & -g/2 & -g/2 \\
            -g/2 & -g/2 & 0 & 6-g & -g/2 & -g/2 \\
            -g/2 & 0 & -g/2 & -g/2 & 8-g & -g/2 \\
            0 & -g/2 & -g/2 & -g/2 & -g/2 & 10-g
            \end{pmatrix}
    \]
    
\end{frame}

\begin{frame}{FCI Results}
    \textbf{Results from Midterm (Pairing Model)}
    Exact diagonalization of the Hamiltonian matrix for $N=4$ particles.
    
    \begin{table}[]
        \centering
        \begin{tabular}{c c c}
        \toprule
        Coupling $g$ & $E_{FCI}$ (Ground State) & Character \\
        \midrule
        -1.0 & 2.7799 & Repulsive \\
        0.0 & 2.0000 & Non-interacting \\
        0.5 & 1.4168 & Attractive \\
        1.0 & 0.6355 & Strong Coupling \\
        \bottomrule
        \end{tabular}
        \caption{Exact eigenvalues computed via \texttt{numpy.linalg.eigh} (from Calculations.ipynb).}
    \end{table}
\end{frame}



%===========================================================
\section{Hartree-Fock Theory}
%===========================================================

\begin{frame}{The Mean-Field}
    \textbf{The Physical Intuition}
    We simplify the many-body problem by assuming each particle moves independently in an average potential (mean field) created by all other particles.
    \vspace{0.3cm}

    \textbf{The Ansatz}
    We assume the ground state is accurately described by a \textbf{single} Slater Determinant (the reference state defined earlier):
    $$ \ket{\Psi} \approx \ket{\Phi_{HF}} $$
    
    The goal of Hartree-Fock is to find the specific set of orthonormal orbitals $\{\phi_p\}$ that minimize the energy of this determinant.
\end{frame}

\begin{frame}{The Energy Functional}
    \textbf{The Quantity to Minimize}
    Before finding the orbitals, we must define the energy of our ansatz. Using the Slater-Condon rules, the expectation value of the Hamiltonian for a single determinant is:
    
    $$ E[\Phi_{HF}] = \bra{\Phi_{HF}} \hat{H} \ket{\Phi_{HF}} $$
    
    \textbf{The Explicit Functional:}
    $$ E_{HF} = \sum_{i=1}^N \bra{i}\hat{h}_0\ket{i} + \frac{1}{2}\sum_{i,j=1}^N \bra{ij}\hat{v}\ket{ij}_{AS} $$
    
    We must find the expansion coefficients $C_{i\alpha}$ (where $\psi_i = \sum C_{i\alpha} \phi_\alpha$) that minimize this scalar $E_{HF}$, subject to the orthonormality constraint $\braket{\psi_i}{\psi_j} = \delta_{ij}$.

\end{frame}

\begin{frame}{The Hartree-Fock Equations}
    \textbf{The Minimization}
    By introducing Lagrange multipliers $\epsilon_i$ to enforce the constraints and setting the derivative $\frac{\delta \mathcal{L}}{\delta C_{i\alpha}} = 0$, we convert the minimization problem into a Generalized Eigenvalue Problem:
    
    \begin{equation}
        \epsilon_i^{HF} C_{i\alpha} = \sum_{\beta} \hat{h}_{\alpha \beta}^{HF}C_{i\beta}
    \end{equation}
    
    \vspace{0.2cm}
    \textbf{The Fock Matrix Elements $\hat{h}_{\alpha \beta}^{HF}$:}
    This matrix effectively "absorbs" the two-body interaction into a one-body mean field:
    $$ \hat{h}_{\alpha \beta}^{HF} = \bra{\alpha}\hat{h}_0\ket{\beta} + \sum_{j \le F} \sum_{\gamma \delta} C_{j \gamma}^* C_{j \delta} \bra{\alpha \gamma} \hat{v} \ket{\beta \delta}_{AS}$$
    
    
\end{frame}


\begin{frame}{Hartree-Fock Results (Pairing Model)}
    \textbf{Application to the Pairing Hamiltonian}
    For the Pairing Model, we found that the standard canonical basis is already the optimal HF basis.
    
    \begin{itemize}
        \item \textbf{Why?} The Pairing Hamiltonian $\hat{V}$ moves pairs ($p \leftrightarrow q$) but does not mix the single-particle levels ($p$).
        \item \textbf{Result:} The Hartree-Fock energy is simply the expectation value of the reference state:
    \end{itemize}
    
    \begin{equation}
        E_{HF} = \bra{\Phi_0} \hat{H} \ket{\Phi_0} = 2 - g
    \end{equation}
    
\end{frame}

\begin{frame}{The Normal-Ordered Hamiltonian}
    \textbf{Calculating Corrections}
    Since Hartree-Fock provides the optimal reference $\ket{\Phi_0}$, it is convenient to measure all energies relative to the HF energy ($E_{HF}$).
    
    We utilize Normal Ordering (Wick's Theorem) to redefine our Hamiltonian:
    
    $$ \hat{H} = E_{HF} + \hat{H}_N $$
    
    \textbf{The Normal-Ordered Hamiltonian $\hat{H}_N$:}
    This operator handles the remaining electron correlation. It has a zero vacuum expectation value ($\bra{\Phi_0} \hat{H}_N \ket{\Phi_0} = 0$).
    
    $$ \hat{H}_N = \hat{F}_N + \hat{V}_N $$
    
    \begin{itemize}
        \item $\hat{F}_N = \sum \epsilon_p \{a_p^\dagger a_p\}$: The one-body Fock operator.
        \item $\hat{V}_N = \frac{1}{4} \sum \bra{pq}v\ket{rs}_{AS} \{a_p^\dagger a_q^\dagger a_s a_r\}$: The two-body interaction.
    \end{itemize}
\end{frame}


%===========================================================
\section{Many-Body Perturbation Theory (MBPT)}
%===========================================================
\begin{frame}{MBPT: Formalism and Partitioning}
    \textbf{Partitioning the Hamiltonian}
    We use the Normal-Ordered form we just defined:
    $$ \hat{H}_N = \hat{F}_N + \hat{V}_N $$
    
    We identify the parts for perturbation theory:
    \begin{itemize}
        \item \textbf{Reference ($\hat{H}_0$):} The Fock operator $\hat{F}_N$.
        \item \textbf{Perturbation:} The fluctuation potential $\hat{V}_N$.
    \end{itemize}

\textbf{The Eigenvalue Problems}
    We seek to estimate the correlation energy $\Delta E$, where:
    \begin{align*}
        \hat{F}_N \ket{\Phi_0} &= W_0 \ket{\Phi_0} \quad (\text{in normal ordering }W_0 = 0) \\
        (E_{HF}+\hat{H}_N) \ket{\Psi} &= (E_{HF} + \Delta E) \ket{\Psi}
    \end{align*}

    \textbf{The Projection Operator $\hat{Q}$}
    We define a projector onto the excited determinants (orthogonal to $\ket{\Phi_0}$):
    $$ \hat{Q} = 1 - \ket{\Phi_0}\bra{\Phi_0} = \sum_{j,a} \ket{\Phi_j^a}\bra{\Phi_j^a} + \dots $$
\end{frame}

\begin{frame}{Formalism of MBPT (Rayleigh-Schrödinger)}
    \textbf{The Energy Expansion}
    Starting from the Schrödinger equation, we derive the iterative expansion for the correlation energy $\Delta E$:

    \begin{equation}
        \Delta E = \sum_{i=0}^{\infty} \bra{\Phi_0} \hat{V}_N \left\{ \frac{\hat{Q}}{W_0 - \hat{F}_N} (\hat{V}_N - \Delta E) \right\}^i \ket{\Phi_0}
    \end{equation}

    \vspace{0.3cm}
\textbf{From Here to Rayleigh-Schrödinger (RS):}
    This equation contains $\Delta E$ on the right-hand side, which generates "unlinked" terms.
    \begin{itemize}
        \item \textbf{The Linked Diagram Theorem:} Proves that the terms containing $\Delta E$ exactly cancel the unlinked diagrams.
        \item \textbf{Final RS Result:} We are left with a series containing only \textit{linked} diagrams, with no $\Delta E$ on the right-hand side.
    \end{itemize}
\end{frame}
\begin{frame}{The Linked Diagram Theorem}

    \textbf{Theorem:}
        The exact energy shift $\Delta E$ is determined \textbf{only} by the sum of all \textbf{linked} diagrams:
        $$ \Delta E = \sum_{n=1}^{\infty} \bra{\Phi_0} \hat{V}_N \left( \frac{\hat{Q}}{W_0 - \hat{H}_0} \hat{V}_N \right)^{n-1} \ket{\Phi_0}_{linked} $$
\vspace{0.5cm}
    \textit{Allows us to ignore disconnected graphs.}
\end{frame}

\begin{frame}{Diagrammatic Representation}
    \textbf{Visualizing Interactions (Goldstone Diagrams)}
    Instead of writing out long strings of operators, we use diagrams to represent contributions to the energy.
    
    \begin{itemize}
        \item \textbf{Time:} Runs upwards.
        \item \textbf{Lines:} 
            \begin{itemize}
                \item Up arrow ($\uparrow$): Particle (state above Fermi level).
                \item Down arrow ($\downarrow$): Hole (state below Fermi level).
            \end{itemize}
        \item \textbf{Vertices:} Dashed horizontal lines represent the interaction $\bra{pq}\hat{v}\ket{rs}$.
    \end{itemize}
    
    \vspace{0.2cm}
    \textit{Example: The Second Order Energy $E^{(2)}$ corresponds to a diagram where two particles are excited (2p-2h), interact, and then de-excite back to the ground state.}
\end{frame}

\begin{frame}
    \begin{figure}
    \centering
    \includegraphics[width=0.6\textwidth]{diagrams.png}
    \caption{Examples of different Diagrams representing perturbative contributions.}
    \label{fig:Diagrams}
\end{figure}
\end{frame}



\begin{frame}{MBPT vs FCI Results}
    \textbf{Convergence Analysis Pairing Model}
    
    \begin{table}[]
        \small
        \centering
        \begin{tabular}{l c c c c}
        \toprule
        Method & $g = -1.0$ & $g = 0.5$ & $g = 1.0$ \\
        \midrule
        $E_{FCI}$ (Exact) & \textbf{2.7799} & \textbf{1.4168} & \textbf{0.6355} \\
        \hline
        $E^{(0+1)}$ (HF) & 3.0000 & 1.5000 & 1.0000 \\
        $E^{(2)}$ & -0.4667 & -0.0624 & -0.2190 \\
        $E^{(3)}$ & +0.5156 & -0.0165 & -0.1005 \\
        $E^{(4)}$ & -0.7527 & -0.0056 & -0.0588 \\
        \hline
        $E_{Total}$ (order 4) & 2.2962 & \textbf{1.4155} & \textbf{0.6217} \\
        \bottomrule
        \end{tabular}
        \caption{Comparison of Perturbation theory orders vs Exact result.}
    \end{table}

    \textbf{Observation:} 
    \begin{itemize}
        \item Excellent agreement for $g=0.5$ (weak coupling).
        \item Divergence/oscillations start appearing at strong coupling ($g=-1.0$).
    \end{itemize}
\end{frame}





%===========================================================
\section{Coupled Cluster Theory}
%===========================================================

\begin{frame}{Coupled Cluster Theory}
    \textbf{The Exponential Wavefunction}
    We define the wavefunction using an exponential operator acting on the reference:
    $$ \ket{\Psi_{CC}} = e^{\hat{T}}\ket{\Phi_0} = \left( 1 + \hat{T} + \frac{1}{2!}\hat{T}^2 + \dots \right) \ket{\Phi_0} $$
    \textit{This ensures size-extensivity (energy additivity) for non-interacting systems.}

    \vspace{0.3cm}
    \textbf{The Cluster Operator}
    The operator $\hat{T}$ generates particle-hole excitations: $\hat{T} = \hat{T}_1 + \hat{T}_2 + \dots$
    
    

    \textbf{Truncation}
    Exact CC is too expensive, so we truncate $\hat{T}$.
    \begin{itemize}
        \item \textbf{CCSD:} $\hat{T} \approx \hat{T}_1 + \hat{T}_2$. The "gold standard" ($O(N^6)$).
        \item \textit{Pairing Model:} The Hamiltonian conserves electron pairs. Single excitations ($\hat{T}_1$) break pairs and are forbidden. I.e. we use \textbf{CCD} ($\hat{T} \approx \hat{T}_2$) without loss of accuracy.
    \end{itemize}
    $$ \hat{T}_{CCD} = \frac{1}{4} \sum_{ijab} t_{ij}^{ab} a_a^\dagger a_b^\dagger a_j a_i $$
\end{frame}

\begin{frame}{CC Equations (Normal Ordered)}
    \textbf{Calculation} To calculate the Normal-Ordered Hamiltonian ($H_N$) we use the BCH expansion:
    $$ \bar{H}_N = e^{-\hat{T}} \hat{H}_N e^{\hat{T}} = \hat{H}_N + [\hat{H}_N, \hat{T}] + \frac{1}{2}[[\hat{H}_N, \hat{T}], \hat{T}] + \dots $$

    \vspace{0.2cm}
    \textbf{Equations} (General CCSD)
    We project $\bar{H}_N \ket{\Phi_0}$ onto the reference and excited determinants, and solve for the equations:
\begin{align*}
\Delta E_{CC} &= \bra{\Phi_0} \bar{H}_N \ket{\Phi_0}\\
0 &= \bra{\Phi_{i}^{a}} \bar{H}_N \ket{\Phi_0}\\
0 &= \bra{\Phi_{ij}^{ab}} \bar{H}_N \ket{\Phi_0}
\end{align*}

    
\end{frame}

\begin{frame}{One Iteration of CCD (Pairing Model)}
    \textbf{The Iterative Process}
    In CCD, we solve for the amplitudes $t_{ij}^{ab}$ iteratively. We start with the MBPT(2) guess and update using the non-linear equations.
    
    \vspace{0.2cm}
    \textbf{Numerical Example ($N=4$, $g=1.0$)}
    
    \begin{itemize}
        \item \textbf{Initialization (MBPT2):} 
        The calculation starts with the perturbative result:
        $$ E_{corr}^{(0)} = -0.2190 $$
        
        \item \textbf{Iteration 1:} 
        Solving the amplitude equations once drastically improves the energy:
        $$ E_{corr}^{(1)} = -0.3128 $$
        
        \item \textbf{Convergence (Iter 60):} 
        The non-linear terms stabilize the solution to the fully correlated limit:
        $$ E_{corr}^{final} = -0.3696 $$
    \end{itemize}
    
\end{frame}


\begin{frame}{Comparison: Pairing Model Results}
    \textbf{CCD vs MBPT vs Exact (FCI)}
    Comparing the correlation energies ($\Delta E = E_{total} - E_{HF}$) across interaction regimes.
    
    \begin{table}[]
        \small
        \centering
        \begin{tabular}{l c c c}
        \toprule
        Method & $g = -1 $ & $g = 0.5$ & $g = 1$ \\
        & (Strong Repulsion) & (Weak) & (Strong Attraction) \\
        \midrule
        \textbf{MBPT(2)} & -0.4667 & -0.0624  & -0.2190 \\
        \textbf{CCD (Iter 60)} & \textbf{-0.2189} & \textbf{-0.0834} & \textbf{-0.3696} \\
        \hline
        \textbf{FCI (Exact)} & \textbf{-0.2201} & \textbf{-0.0832} & \textbf{-0.3645} \\
        \bottomrule
        \end{tabular}
        \caption{Correlation energies calculated for N=4 particles.}
    \end{table}

    \textbf{Conclusions:}
    \begin{itemize}
        \item \textbf{Weak Coupling ($g=0.5$):} All methods agree well.
        \item \textbf{Strong Coupling ($g=1.0$):} MBPT2 underestimates correlation by $\approx 40\%$. CCD captures the infinite order effects, matching Exact FCI almost perfectly.
        \item \textbf{Robustness:} CCD remains stable and accurate even where simple perturbation theory begins to break down.
    \end{itemize}
\end{frame}


%===========================================================
\section{Density Functional Theory (DFT)}
%===========================================================
\begin{frame}{Motivation}
    \textbf{The Problem}
    Solving the Schrödinger equation for $\Psi(\mathbf{r}_1, \dots, \mathbf{r}_N)$ is exponentially expensive.
    

    \vspace{0.3cm}
    \textbf{Solution}
    Replace the wavefunction with the electron density $n(\mathbf{r})$:
    $$ n(\mathbf{r}) = N \int |\Psi(\mathbf{r}, \mathbf{r}_2, \dots, \mathbf{r}_N)|^2 d^3r_2 \dots d^3r_N $$
    \textit{This reduces the problem from $3N$ variables to just 3 ($x, y, z$).}
\end{frame}

\begin{frame}{The Hohenberg-Kohn Theorems}
    \textbf{Theorem I: Existence}
    The external potential $v_{ext}$ is uniquely determined by the ground state density $n_0(\mathbf{r})$.
    $$ n_0(\mathbf{r}) \leftrightarrow v_{ext}(\mathbf{r}) \leftrightarrow \Psi_0 $$

    \textbf{Theorem II: The Universal Functional}
    The total energy is a functional of the density:
    $$ E[n] = \int v_{ext}(\mathbf{r})n(\mathbf{r}) d^3r + F_{HK}[n] $$
    
    Where $F_{HK}[n]$ is a \textbf{universal functional} (independent of the specific system) containing all internal electron operators:
    $$ F_{HK}[n] = \bra{\Psi} \hat{T} + \hat{V}_{ee} \ket{\Psi} $$
\end{frame}

\begin{frame}{The Kohn-Sham Method}
    \textbf{The Ansatz}
    To make the problem solvable, Kohn and Sham mapped the real interacting system to a fictitious system of non-interacting electrons (helps with $T$) that yields the exact same ground state density:
    $$ n(\mathbf{r}) = \sum_{i=1}^{N} |\phi_i(\mathbf{r})|^2 $$

    \vspace{0.2cm}
    \textbf{Partitioning the Functional}
    We rewrite the universal functional $F_{HK}$ by separating the known parts from the unknown many-body effects:
    $$ F_{HK}[n] = T_s[n] + E_{H}[n] + E_{XC}[n] $$
    
    \textbf{The Exchange-Correlation Energy ($E_{XC}$)}
    This term is defined as the "garbage bin" for all the complex physics we cannot calculate exactly (quantum exchange and correlation):
    $$ E_{XC}[n] = (T_{true} - T_s) + (V_{ee} - E_{H}) $$
\end{frame}

\begin{frame}{Implementation: Equations and Approximations}
    \textbf{Deriving the Equations}
    Minimizing the total energy $E[n]$ with respect to the orbitals $\phi_i$ (variational principle) yields a set of single-particle eigenvalue equations:
    $$ \left( -\frac{\hbar^2}{2m}\nabla^2 + v_{ext} + v_{Hartree} + \frac{\delta E_{XC}}{\delta n} \right) \phi_i = \epsilon_i \phi_i $$

    \vspace{0.3cm}
    \textbf{The Unknown Term}
    The equation above is exact, but useless without knowing the functional derivative $\frac{\delta E_{XC}}{\delta n}$. We must approximate it.
    
    \textbf{The Solution: Local Density Approximation (LDA)}
    LDA assumes that at any point $\mathbf{r}$, the exchange-correlation energy depends only on the density at that specific point, using the known values from a Homogeneous Electron Gas ($\epsilon_{HEG}$):
    $$ E_{XC}^{LDA}[n] = \int n(\mathbf{r}) \epsilon_{HEG}(n(\mathbf{r})) d^3r $$
\end{frame}

\begin{frame}{Comparison: DFT vs Hartree-Fock}
    \textbf{Similarities:} Both are Mean-Field theories solving single-particle equations ($O(N^3)$ to $O(N^4)$).

    \vspace{0.3cm}
    \begin{table}[]
        \centering
        \begin{tabular}{l | l | l}
        \toprule
        \textbf{Feature} & \textbf{Hartree-Fock (HF)} & \textbf{DFT (Kohn-Sham)} \\
        \midrule
        \textbf{Exchange} & Exact. & Approximate (in $E_{XC}$). \\
        \hline
        \textbf{Correlation} & None (by definition). & Included (approx. in $E_{XC}$). \\
        \hline
        \textbf{Main Error} & Always overestimates Energy. & Depends on functional choice. \\
        \bottomrule
        \end{tabular}
        \caption{Key differences between the two mean-field approaches.}
    \end{table}
\end{frame}


%===========================================================
\section{Summary}
%===========================================================

\begin{frame}{Final Summary}
    \begin{table}[]
        \centering
        \footnotesize % Ensures text fits comfortably
        \setlength{\tabcolsep}{6pt} % Adds padding inside columns
        \renewcommand{\arraystretch}{1.3} % Adds vertical space between rows
        
        \begin{tabular}{|l|p{4.2cm}|p{4.2cm}|}
            \hline
            \textbf{Method} & \textbf{Strengths} & \textbf{Weaknesses} \\
            \hline
            \textbf{FCI} & \textbf{Exact solution.} The benchmark for all other methods. & Exponential scaling ($N!$). Limited to tiny systems. \\
            \hline
            \textbf{Hartree-Fock} & Computationally cheap ($O(N^4)$). Provides the orbital basis. & Zero correlation energy. \\
            \hline
            \textbf{MBPT} & Systematically improvable. Simple to implement. & Diverges for strong interactions (e.g., $g=-1$). More computationally expensive for every term. \\
            \hline
            \textbf{Coupled Cluster} & The Gold Standard. High accuracy and size-extensive. & Expensive ($O(N^6)$). Mathematical complexity. \\
            \hline
            \textbf{DFT} & Efficient ($O(N^3)$) with electron correlation included. & Exact functional is unknown. No systematic improvement. \\
            \hline
        \end{tabular}
    \end{table}

    \vspace{0.2cm}

        \begin{itemize}
            \item \textbf{Weak Coupling ($g=0.5$):} Perturbation theory (MBPT) is sufficient and efficient.
            \item \textbf{Strong Coupling ($g=-1.0$):} Perturbation theory breaks down. Non-perturbative methods (Coupled Cluster or FCI) are strictly required to capture the physics.
        \end{itemize}
\end{frame}

\begin{frame}
    \centering
    \Huge Thank You!
    
\end{frame}

%===========================================================
\section{appendix}
%===========================================================
\begin{frame}{The Lipkin-Meshkov-Glick (LMG) Model}
    \textbf{System Definition}
    \begin{itemize}
        \item $N=4$ fermions distributed in two energy levels with degeneracy $d=4$.
        \item States characterized by spin $\sigma = \pm 1$ and quantum number $p$.
    \end{itemize}

    \textbf{Quasi-Spin Formalism}
    The Hamiltonian is expressed using quasi-spin operators $\hat{J}_z, \hat{J}_\pm$:
    \begin{equation}
        \hat{H} = \hat{H}_0 + \hat{H}_1 + \hat{H}_2
    \end{equation}
    
    Using the commutation relations derived in Midterm 1, these terms are:
    \begin{align*}
        \hat{H}_0 &= \epsilon \hat{J}_z \\
        \hat{H}_1 &= \frac{1}{2}V (\hat{J}_+^2 + \hat{J}_-^2) \quad \text{(Moves pairs of particles)} \\
        \hat{H}_2 &= \frac{1}{2}W (\hat{J}_+ \hat{J}_- + \hat{J}_- \hat{J}_+ - \hat{N}) \quad \text{(Spin exchange term)}
    \end{align*}
\end{frame}

\begin{frame}{Hartree-Fock Stability in the Lipkin Model}
    \textbf{Variational Analysis}
    We introduced a mixing coefficient $x$ ($C_{\alpha+}^2 = x$) between the levels. The stability of the non-interacting ground state ($x=0$) depends on the interaction strength.

    \begin{block}{Stability Condition}
        The trivial HF solution is a stable minimum if and only if:
        $$ \frac{\epsilon}{3} > -(V + W) $$
    \end{block}

    \textbf{Phase Transition Example}
    \begin{itemize}
        \item \textbf{Case 1 - Weak:} $\frac{\epsilon}{3} > -(V+W) \implies$ Standard HF is stable.
        \item \textbf{Case 2 - Strong:} With $V=-4/3, W=-1$, the condition fails ($0.67 < 2.33$). 
        \item \textbf{Result:} The system undergoes a phase transition to a "mixed" state with lower energy ($E_{mixed} \approx -7.57$ vs $E_{HF} = -4.0$) which is much closer to the FCI result of $-7.75$.
    \end{itemize}
\end{frame}

\begin{frame}{Second Order MBPT Energy Pairing model}
    \begin{figure}
    \centering
    \includegraphics[width=0.6\textwidth]{MBPT2.png}
    \caption{The two diagrams for second order energy in MBPT}
    \label{fig:MBPT2}
\end{figure}
In our case we have:
    \[
    E^{(2)} = \sum_{a > F} \sum_{i \leq F} \frac{\bra{i\bar{i}}V_N\ket{a\bar{a}}| \bra{a\bar{a}}V_N\ket{i\bar{i}}}{2(\epsilon_i - \epsilon_a)}
    \]
Where $E^{(2)}$ scales as $O(N^4)$.
\end{frame}



\end{document}