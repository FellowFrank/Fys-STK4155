\documentclass[11pt, a4paper]{article}

% --- Packages ---
\usepackage[utf8]{inputenc}
\usepackage[T1]{fontenc}
\usepackage{geometry}       % Controls margins
\usepackage{enumitem}       % Custom lists for the timeline
\usepackage{hyperref}       % Hyperlinks
\usepackage{parskip}        % Adds space between paragraphs instead of indenting
\usepackage{amsmath}

% --- Formatting Setup ---
\geometry{left=2.5cm, right=2.5cm, top=2.5cm, bottom=2.5cm}
\hypersetup{colorlinks=true, linkcolor=blue, urlcolor=blue}

% --- Document Info ---
\title{Thesis Proposal: Quantum Algorithms for Partial Differential Equations}
\author{
    \textbf{Student:} Anton N. Torgersen\\
    \textbf{Supervisor:} Franz Georg Fuchs\\
    \textbf{Internal Supervisor:} Morten Hjorth-Jensen} % Replace with your name
\date{\today}

\begin{document}

\maketitle

% --- Thesis Statement ---
\section*{Thesis Statement}
This thesis investigates the computational complexity and practical implementation of high-precision quantum algorithms for solving linear partial differential equations (PDEs). Specifically, it aims to benchmark the exponential improvements in error dependence ($\text{polylog}(1/\epsilon)$) proposed by modern spectral methods against standard classical solvers.

% --- Background Section (Incorporating Sources) ---
\section*{Background \& Motivation}
Partial differential equations are ubiquitous in physics and engineering, but classical grid-based methods suffer from the "curse of dimensionality." While early quantum approaches (such as the HHL algorithm) offered potential speedups, they were limited by a polynomial dependence on the inverse error, denoted as $O(\text{poly}(1/\epsilon))$. This made high-precision solutions prohibitively expensive for near-term applications.

Recent literature, including the work of Childs et al. (2020) \cite{childs2020} and extensive analysis from the University of Waterloo \cite{waterloo_thesis}, introduces a new class of algorithms utilizing spectral methods and Hamiltonian simulation. These methods theoretically allow for time evolution where the complexity scales logarithmically with the inverse error, $O(\text{polylog}(1/\epsilon))$. This project seeks to verify these bounds and explore the feasibility of these algorithms using current quantum simulation tools.

% --- Methodology ---
\section*{Methodology}
The project will be conducted through a combination of theoretical analysis and numerical simulation:
\begin{itemize}
    \item \textbf{Literature Review:} A comparative study of classical finite difference methods versus quantum spectral methods, specifically focusing on the complexity proofs provided in \cite{childs2020}.
    \item \textbf{Implementation:} Implementation of toy-model PDEs (e.g., heat or wave equation) using quantum SDKs such as Qiskit or PennyLane to simulate Hamiltonian evolution.
    \item \textbf{Benchmarking:} Assessing the gate depth and resource requirements of the quantum approach relative to the desired precision $\epsilon$.
\end{itemize}


\section*{Progression Plan \& Milestones}

\emph{Note: Weekly meetings between supervisor and student will be held from the start of Autumn 2026 until the thesis is finished. With some preliminary meetings in Spring 2026.}

\begin{description}[style=multiline, leftmargin=4cm, font=\bfseries]

    \item[Spring 2026] 
    \textbf{Preparation Phase} \\
    Completion of coursework (MAT 4430 and FYS 4415). Initial meetings regarding the topic and discussion of relevant research papers.

    \item[Autumn 2026 (Q3)] 
    \textbf{Theoretical Exploration} \\
    Theoretical exploration of PDEs and Quantum algorithms, specifically analyzing their strengths and weaknesses.

    \item[Autumn 2026 (Q4)] 
    \textbf{Specialisation \& Computation} \\
    Specialisation of the research to a specific result or area within the theory. Execution of numerical computations and comparisons.

    \item[Spring 2027 (Q1)] 
    \textbf{Deep Dive} \\
    Finishing the deep dive in the specific area and resolving theoretical loose ends.

    \item[Spring 2027 (Q2)] 
    \textbf{Writing \& Compilation} \\
    Compiling and rewriting the results derived from the previous three quarters into a consistent whole.

\end{description}

\section*{Relevant Curriculum}
\begin{itemize}
    \item \textbf{MAT 4430} -- Quantum Information Theory
    \item \textbf{FYS 4415} -- Quantum Computing and Quantum Information
\end{itemize}

\begin{thebibliography}{9}

\bibitem{childs2020}
A. M. Childs, J. P. Liu, and A. Ostrander, 
``High-precision quantum algorithms for partial differential equations,'' 
\textit{Quantum}, vol. 5, p. 574, 2021. [Online]. Available: arXiv:2002.07868.

\bibitem{waterloo_thesis}
D. An, 
``Quantum algorithms for partial differential equations,'' 
Ph.D. dissertation, Univ. Waterloo, Waterloo, ON, Canada, 2022. [Online]. Available: UWSpace.

\end{thebibliography}


\end{document}